\documentclass[12pt]{article}
\usepackage{amsmath}
\usepackage[hmargin=1in,vmargin=1in]{geometry}
\usepackage{amsfonts}
\usepackage{amssymb}
\usepackage{fancyhdr}
\usepackage{lastpage}
\usepackage{parskip} % Kill first paragraph indenting
\usepackage{tikz}
\usepackage[T1]{fontenc}
\usepackage{environ}



\pagestyle{fancy}
\fancyhead[L]{Omeed Mariani \\ CMSC/Math 207}
\fancyhead[R]{HW 5 \\ Page \thepage\ of \pageref{LastPage}}

\newcounter{problemnum}
\newcommand{\newprob}{\addtocounter{problemnum}{1} \noindent (\textbf{\arabic{problemnum}) }}

\begin{document}

\newprob Use the Quotient Remainder Theorem with $d$ = 3 to prove that the product of any two
consecutive integers has the form $3$$k$ or $3$$k$ + 2.

Let $n \in \mathbb{Z }$,\space The Quotient Remainder Theorem tells us that every integer is one of the form 3$m$, 3$m$+1, or 3$m$+2.
\newline
\newline
[Case 1] $n=3m$, then the consecutive integers are of the form 3$m$, 3$m$+1.The product should be $n(n+1)=3m(3m+1)$, this then equals 3$k$ for some integer $k=(3m+1)$.
\newline
\newline
[Case 2] $n$=3$m$+1 If the consecutive integers are 3$m$+1 and 3$m$+2 then the product is
\begin{align*}
n(n+1)
&=(3m+1)(3m+2)\\
&= 3(3m^2 +3m)+2. \\
\end{align*}

\newline
\newline
[Case 3] $n=3m+2$ if the consecutive integers are of the form 3$m$+2 and 3$m$+3, then the product is of the form
\begin{align*}
n(n+1)
&=(3m+2)(3m+3) \\
&=3(3m^2+5m+2).  \\
\end{align*}
From these three cases we can confirm that the product of any two integers is of the form 3$k$ or 3$k$+2.

\newprob \newline
\newline
\textbf{(a)} The standard factored form of $a$$^3$ would be the standard form of $a$ cubed.\newline
\begin{align*}
a^3
&=(p_1^{e_1}\cdot p_2^{e_2}...p_k^{e_k})^3,\\
a^3
&= p_1^{3e_1}\cdotp p_2^{3e_2}...p_k^{3e_k}.
\end{align*}


\textbf{(b)} To start, we must find the lowest positive integer $k$, such that $2^4\cdot3^5\cdot7\cdot11^2\cdot k$
is a perfect cube, or in other words the integer must equal the third power. Let's take the least positive integer $k$ =  2$^4$\cdot3$^5$\cdot7\cdot11$^2$.
\begin{align*}
2^4\cdot3^5\cdot7\cdot11^2\cdotk  &=2^4\cdot3^5\cdot7\cdot11^2\cdot2^2\cdot3^1\cdot7^2\cdot11\\
&=2^6\cdot3^6\cdot11^3\cdot7^3\\
&=(2^2)^3\cdot(3^2)^3\cdot11^3\cdot7^3\\
&=(2^2\cdot3^2\cdot11\cdot7)^3\\
&=(2772)^3.\\
\end{align*}
\text{By definition of a perfect cube,}
\begin{align*}
k
&=2^2\cdot3^1\cdot7^2\cdot11\\
&=4\cdot3\cdot49\cdot11.
\end{align*}
Finally, the least positive integer
$k = 6468$


































\end{document}
