\documentclass[12pt]{article}
\usepackage{amsmath}
\usepackage[hmargin=1in,vmargin=1in]{geometry}
\usepackage{amsfonts}
\usepackage{amssymb}
\usepackage{fancyhdr}
\usepackage{lastpage}
\usepackage{parskip} % Kill first paragraph indenting
\usepackage{tikz}
\usepackage[T1]{fontenc}
\usepackage{environ}



\pagestyle{fancy}
\fancyhead[L]{Omeed Mariani \\ CMSC/Math 207}
\fancyhead[R]{HW 6 \\ Page \thepage\ of \pageref{LastPage}}

\newcounter{problemnum}
\newcommand{\newprob}{\addtocounter{problemnum}{1} \noindent (\textbf{\arabic{problemnum}) }}

\begin{document}

\newprob Prove that if $a$ and $b$ are rational numbers $b$ $\neq 0$, and $r$ is an irrational number, then $a + br$ is irrational.\\
\\
Let $a,b\in \mathbb{Q}$ and $r \in \mathbb{I}$. There are integers $m$, $n$ where $n$ $\neq 0$, $a=\frac{m}{n}$.Then $b=\frac{s}{t}$, $t$ $\neq 0$, for some intgers $s$ and $t$.
We know that $a+br$ is rational, where $a+br=\frac{p}{q}$, for some integers $p$ and $q$, $q$ $\neq 0$.
\\
We can consider
\begin{align*}
a+br
&=\frac{p}{q}\\
\frac{m}{n}+\left(\frac{s}{t}\right)\cdot r
&=\frac{p}{q}\\
r&=(\frac{p}{q} - \frac{m}{n}) \cdot \frac{t}{s}\\
&=\frac{(t(np-mq))}{qns}.
\end{align*}

Where $n$ $\neq 0$, $q$ $\neq 0$, $b$ $\neq 0$ therefore, $qns \neq 0$ using the zero product property.
The integer $r$ is over a non integer, proving that $r$ is a rational number, contradicting the fact that $r$ is irrational and also proving $a+br$ is irrational.

\newprob Prove that for every integer $a$, if $a^3$ is even then $a$ is even.\\
\\
Let  $a$ be an odd integer. Then $a=2k+1$ for some $k \in \mathbb{Z}$.
\begin{align*}
a^3
&=(2k+1)^3\\
&=8k^3+12k^2+6k+1\\
&=2(4k^3+6k^2+3k)+1.\\
\end{align*}
Since $k \in \mathbb{Z}$, $4k^3+6k^2+3k$ $\in  \mathbb{Z}$. Thus $a^3$ is odd.

\newprob Prove that $\sqrt[3]{2}$ is irrational.\\
We can do a proof by contradiction. To start, we can assume $\sqrt[3]{2}$ is rational. Let $\sqrt[3]{2}
&= p/q$ \text{where } $p$ \text{ and } $q$ \text{ are integers with no common divisors}\\
\begin{align*}
2
&=(p/q)^3\\
p^3
&=2q^3
\end{align*}
Here, $p^3$ is even by problem 2. Let $p = 2k$ for some integer $k$. Then,we subsitute $p=2k$ in $(p/q)^3=2$.
\begin{align*}
(p/q)^3
&=2\\
p^3
&=2q^3\\
(2k)^3
&={2q^3}\\
4k^3
&=q^3\\
\end{align*}
Hence, $q^3$ is even, therefore $q$ is even by number 2. Then $p$ and $q$ are divisible by 2 and hence, $p$ and $q$ have the common factor of 2. This is a contradiction. Therefore, $a=\sqrt[3]{2}$ is an irrational number. \\
\newprob Consider the following statement:
\begin{align*}
    (\forall m, n\in \mathbb{Z}) \text{ if } 2m+n \text{ is odd then } m \text{ and } n \text{ are both odd. }
\end{align*}
\\
\textbf{(a)}\\
$2m+n$ is odd implies that an even number multiplied by $2m$ then added to 1 equals an odd number. If we assume $n$ is an odd number, $2m+n$ will still result to be an odd number. For example, we can use $4$ as $m$ and $3$ as $n$.
\begin{align*}
2m+n
&=8+3\\
&=11,
\end{align*}
\text{ which is odd.}\\
\textbf{(b)}\\
The negation of the statement is :
\begin{align*}
    \exists \text{ integers } m \text{ and } n, 2m+n \text{ is odd } \text{and both } m \text{ and } n \text{ are not odd }
\end{align*}

\newprob Determine, with proof, whether this statement is true or false: The product of any two
irrational numbers is irrational.

We can have $9+2\sqrt3$ be an irrational number and $9-2\sqrt3$ be another irrational number.If we were to multiply $9+2\sqrt3$ and $9-2\sqrt3$ we then get our answer:
\begin{align*}
(9+2\sqrt3)(9-2\sqrt3)
&=81-4(3)\\
&=81-12\\
&=69,
\end{align*}
   Which is rational. This is a contradiction to the statement 'The product of any two irrational numbers is irrational.' Therefore the statement is false.





































\end{document}
