\documentclass[12pt]{article}
\usepackage{amsmath}
\usepackage[hmargin=1in,vmargin=1in]{geometry}
\usepackage{amsfonts}
\usepackage{amssymb}
\usepackage{fancyhdr}
\usepackage{lastpage}
\usepackage{parskip} % Kill first paragraph indenting
\usepackage{tikz}
\usepackage[T1]{fontenc}
\usepackage{environ}



\pagestyle{fancy}
\fancyhead[L]{Omeed Mariani \\ CMSC/Math 207}
\fancyhead[R]{HW 8 \\ Page \thepage\ of \pageref{LastPage}}

\newcounter{problemnum}
\newcommand{\newprob}{\addtocounter{problemnum}{1} \noindent (\textbf{\arabic{problemnum}) }}

\begin{document}

\newprob Prove that $3^{2n}-1$ is divisible by 8 for each integer $n 	\geq 0$.\\
\\
For $n \in \mathbb{N}$, let $P(n)$ be divisible by 8 for each integer $n \geq 0$.\\
\textbf{Basis:} \\
3^{2(0)}-1=1-1=0$, which is divisible by 8.\\
\textbf{Induction:}\\
Let $n \in \mathbb{N}$, and assume $P(n)$ is true, that is, $3^{2n}-1$ is divisible by 8. Thus, by the definition of divisibility, there exists an integer $p$, such that $3^{2k}-1=8p$\\
Now we must prove that $P(n+1)$ is true. \\
Consider:
\begin{align*}
3^{2(n+1)}-1
&=3^{2n+2}-1\\
&=3^{2n} \cdot 3^2-1\\
&=9(8p+1)-1\\
&=72p+9-1\\
&=72p+8\\
&=8(9p+1)\\
\end{align*}
$3^{2n+1}-1$ is divisible by $8$. Thus, $3^{2n}-1$ is divisible by 8 for each integer $n 	\geq 0$, and the value of $(9p+1)$ is also an integer.\\
\\
\newprob Prove that $1+4n<2^n$ for all integers $n 	\geq 5$.\\
For $n \in \mathbb{W}$, assume $P(n)$ is greater than or equal to 5.\\
\textbf{Basis:}\\
\begin{align*}
&1+4(5)<2^5\\
\end{align*}
\\
$\therefore P(5)$ is true. \\
Let $n \in \mathbb{N}, n \geq 5$, and assume P(n) is true\\
\textbf{Induction:}\\
\begin{align*}
1+4n+4
&<2^n+2^n\\
1+4(n+1)
&<2^n(1+1)=2^n\cdot 2=2^{n+1}\\
1+4(n+1)
&<2^{n+1}
\end{align*}
$\therefore 1+4n<2^n$ is true for all integers $n 	\geq 5$.




































\end{document}
